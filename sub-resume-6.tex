\expandafter\ifx\csname readornot\endcsname\relax
  \documentclass[uplatex]{jsarticle}
  \usepackage{octopus}
  \usepackage{url}

  \renewcommand{\proofname}{\textsf{証明}}
  \renewcommand{\postpartname}{章}
  \renewcommand{\thesection}{\thepart.\arabic{section}}
  \renewcommand{\thepart}{\arabic{part}}
  \makeatletter\renewcommand{\theequation}{\thesection.\arabic{equation}}\@addtoreset{equation}{section}\makeatother

  \newcommand{\octopuspart}[1]{\newpage\part{#1}\setcounter{section}{0}\vspace{3\baselineskip}}

  \DeclareMathOperator{\dcup}{\dot{\cup}}
  \begin{document}
\fi

\section{基本群}

\midashi{これからの展望}

\begin{center}
  \begin{tikzpicture}
    \draw (0,0) circle [x radius=1, y radius=0.5] node {空間 $X$};
    \draw (0,-2) circle [x radius=1, y radius=0.5] node {空間 $Y$};
    \node at (3.5,0) {群$\pi_{1}(X)$};
    \node at (3.5,-2) {群$\pi_{1}(Y)$};
    \node[left] at (0,-1) {ホモトピー同値\hspace{1zw}\rotatebox[origin=c]{-90}{$\simeq$}};
    \node[right] at (3.5,-1) {\rotatebox[origin=c]{-90}{$\simeq$}\hspace{1zw}群同型};
    \draw[->] (1.5,0) -- (2.5,0);
    \draw[->] (1.5,-2) -- (2.5,-2);
  \end{tikzpicture}
\end{center}
となるような空間と群の対応を構成したい。

\sukima \midashi{基本群}

$X$:位相空間
\begin{teigi}[パス]
  \nw{パス(path)} $\defines$ $f \colon [0,1] \longrightarrow X$:連続
\end{teigi}

\begin{teigi}[ホモトピー]
  $\sets{f_{t}}$:パスの\nw{ホモトピー} $\defines$ 次の条件を満たす$\sets{f_{t} \colon [0,1] \longrightarrow X}_{t \in [0,1]}$のこと:
  \begin{itemize}
    \vspace{-0.5\baselineskip}
    \item $\forall t \in [0,1],\quad f_{t}(0) = x_{0},\quad f_{t}(1) = x_{1}$
    \item $F \colon \mapdef{[0,1] \times [0,1]}{X}{(s,t)}{f_{t}(s)}$が連続
  \end{itemize}
\end{teigi}

\begin{teigi}[ホモトープ]
  2つのパス$f', f''$が\nw{ホモトープ} $\defines$ $f', f''$をつなぐホモトピー$\sets{f_{t}}$であって,
  $f_{0} = f'$,$f_{1} = f''$であるものが存在する

  このことを$f' \simeq f''$と書く。
\end{teigi}

\begin{prop}
  ホモトープの$\simeq$は同値関係である。
\end{prop}

\begin{proof}
  パスのホモトープは普通のホモトープの特別な場合であるから,そのときの同値性から従う。
\end{proof}

\begin{teigi}[パスの積(合成)]
  $f,g \colon [0,1] \longrightarrow X$:パス,$f(1) = g(0)$を満たすとする。
  パスの\nw{合成} $f \cdot g \colon [0,1] \longrightarrow X$を次で定義する。
  \begin{equation}
    f \cdot g (s) := \begin{cases}
      f(2s) & \left( 0 \le s \le \dfrac{1}{2} \right) \\
      g(2s-1) & \left( \dfrac{1}{2} \le s \le 1 \right)
    \end{cases}
  \end{equation}
\end{teigi}

\sukima \midashi{注意.} $f' \simeq f$,$g' \simeq g$ならば$f' \cdot g' \simeq f \cdot g$である。

なぜならば,$f$と$f'$を結ぶホモトピー$\sets{f_{t}}$,$g$と$g'$を結ぶホモトピー$\sets{g_{t}}$に対して
$\left( f \cdot g \right)_{t} := f_{t} \cdot g_{t}$とすれば,これが$f' \cdot g'$と$f \cdot g$を結ぶホモトピーになるから。

\begin{teigi}[ループ]
  $x_{0}$を\nw{基点(basepoint)とするループ(loop)} $\defines$ パス$f \colon [0,1] \longrightarrow X$であって,$f(0) = f(1) = x_{0}$
\end{teigi}

\expandafter\ifx\csname readornot\endcsname\relax
  \end{document}
\fi