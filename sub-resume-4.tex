\expandafter\ifx\csname readornot\endcsname\relax
  \documentclass[uplatex]{jsarticle}
  \usepackage{octopus}
  \usepackage{url}

  \renewcommand{\proofname}{\textsf{証明}}
  \renewcommand{\postpartname}{章}
  \renewcommand{\thesection}{\thepart.\arabic{section}}
  \renewcommand{\thepart}{\arabic{part}}
  \makeatletter\renewcommand{\theequation}{\thesection.\arabic{equation}}\@addtoreset{equation}{section}\makeatother

  \newcommand{\octopuspart}[1]{\newpage\part{#1}\setcounter{section}{0}\vspace{3\baselineskip}}

  \DeclareMathOperator{\dcup}{\dot{\cup}}
  \begin{document}
\fi

\section{コンパクト性}
$(X, \mathcal{O})$:位相空間,$A \subseteq X$とする。

\begin{teigi}[被覆]
  $\mathcal{C} \subseteq 2^{X}$:$A$の\nw{被覆} $\defines$ ${\displaystyle A \subseteq \bigcup_{C \in \mathcal{C}} C}$

  特に,$\mathcal{C} \subseteq \mathcal{O}$のとき,$\mathcal{C}$を\nw{開被覆}という。
\end{teigi}

\begin{teigi}[コンパクト]
  \midashi{(1)} $A \subseteq X$:\nw{コンパクト} $\defines$ $\forall \mathcal{C}$:$A$の開被覆,$\exists O_{1}, \dots, O_{k} \in \mathcal{C}$,${\displaystyle A \subseteq \bigcup_{j=1}^{k} O_{j}}$

  \midashi{(2)} $X$がコンパクトのとき,$(X, \mathcal{O})$をコンパクト空間という。
\end{teigi}

$A \subseteq X$がコンパクトであることは,標語的に「任意の開被覆は有限部分開被覆を含む」ということができる。
また,$A \subseteq X$がコンパクトであることは,$(A, \mathcal{O}_{A})$がコンパクト空間であることと同値である。
ただし,$\mathcal{O}_{A}$は相対位相。

\begin{hodai}
  $A_{1}, \dots, A_{k} \subseteq X$ :コンパクト $\Longrightarrow$ $A_{1} \cup \dots \cup A_{k}$:コンパクト
\end{hodai}

\begin{hodai}
  $(X, \mathcal{O})$:コンパクト $\Longrightarrow$ $A \in \mathfrak{A}$:コンパクト
\end{hodai}

\begin{proof}
  $A$の開被覆と$X \setminus A$が$X$の開被覆になっていることから従う。
\end{proof}

\begin{hodai}
  $X$:コンパクト,$f \colon X \longrightarrow Y$:連続 $\Longrightarrow$ $f(X)$:コンパクト
\end{hodai}

\begin{proof}
  $\mathcal{C}$:$f(X)$の開被覆とすると,$f^{-1}(\mathcal{C}) := \sets{f^{-1} (O) | O \in \mathcal{C}}$は$f$の連続性から$f^{-1}(\mathcal{C}) \subseteq \mathcal{O}$を満たし,
  これは$X$の開被覆である。
  $X$のコンパクト性から,ある$O_{1}, \dots, O_{k} \in \mathcal{C}$が存在して,
  $f^{-1}(O_{1}) \cap \dots \cap f^{-1}(O_{k}) = X$を満たす。
  このとき,$f(X) \subseteq O_{1} \cap \dots \cap O_{k}$である。
\end{proof}

\begin{hodai}
  $(X, d)$:距離空間とする。
  $A \subseteq X$:コンパクト $\Longrightarrow$ $A$は有界な閉集合
\end{hodai}

\begin{proof}
  \midashi{[有界性]} $x \in X$とする。開集合族$\sets{N(x,r) | r \in \mathbb{N}}$は$A$の被覆である。
  したがってコンパクト性から有限個の$r_{1} < r_{2} < \dots < r_{k}$が存在して$\sets{N(x,r_{i})}_{i=1}^{k}$は$A$の開被覆である。
  よって,$A \subseteq N(x,r_{k})$であり,$A$は有界。

  \midashi{[閉性]} $X \setminus A$が開集合であることを示す。$y \in X \setminus A$とする。
  ${\displaystyle \sets{X \setminus \overline{N \left( y, \frac{1}{r} \right)} | r \in \mathbb{N}}}$は$A$の開被覆である。
  したがってコンパクト性から有限個の$r_{1} < r_{2} < \dots < r_{k}$が存在して${\displaystyle \sets{X \setminus \overline{N \left( y,\frac{1}{r_{i}} \right)}}_{i=1}^{k}}$は$A$の開被覆である。
  よって,${\displaystyle A \subseteq X \setminus \overline{N \left( y,\frac{1}{r_{1}} \right)}}$であり,
  ${\displaystyle N \left( y, \frac{1}{r_{1}} \right) \subseteq X \setminus A}$であるから,$X \setminus A$は開集合である。
\end{proof}

\begin{corr}
  $X$:コンパクト,$f \colon X \longrightarrow \mathbb{R}$:連続とする。
  このとき,${\displaystyle \max_{x \in X} f(x)}$,${\displaystyle \min_{x \in X} f(x)}$が存在する。
\end{corr}

\begin{proof}
  $f(X)$はコンパクトであり有界閉集合であるから。
\end{proof}

\midashi{演習} 有界閉集合がコンパクトにならない距離空間の例を挙げよ。

\begin{hodai}[Lebesgue数の補題]
  $(X,d)$:コンパクトな距離空間,$\mathcal{C}$:$X$の開被覆とする。このとき
  \begin{equation}
    \exists \delta > 0, \quad \forall x \in X, \quad \exists O \in \mathcal{C}, \quad N(x, \delta) \subseteq O
  \end{equation}
  を満たす。
\end{hodai}

\begin{proof}
  $X$のコンパクト性から有限な部分開被覆$\sets{O_{i}}_{i=1}^{k} \subseteq \mathcal{C}$が存在する。
  \begin{equation}
    f(x) := \frac{1}{k} \sum_{i=1}^{k} d(x, X \setminus O_{i})
  \end{equation}
  とすると$f$は連続である。被覆性から常に$f(x) > 0$である。
  すなわち${\displaystyle \delta := \min_{x \in X} f(x) > 0}$である。よって,$\forall x \in X, \quad f(x) \ge \delta$である。
  これより,すべての$x \in X$に対してある$i \in \sets{1,\dots,k}$が存在して$d(x, X \setminus O_{i}) \ge \delta$である。
  よって,$N(x, \delta) \subseteq O_{i}$である。
\end{proof}

\expandafter\ifx\csname readornot\endcsname\relax
  \end{document}
\fi