\documentclass[uplatex]{jsarticle}
\usepackage{octopus}
\usepackage{url}

\renewcommand{\proofname}{\textsf{証明}}
\renewcommand{\postpartname}{章}
\renewcommand{\thesection}{\thepart.\arabic{section}}
\renewcommand{\thepart}{\arabic{part}}
\makeatletter\renewcommand{\theequation}{\thesection.\arabic{equation}}\@addtoreset{equation}{section}\makeatother

\newcommand{\octopuspart}[1]{\newpage\part{#1}\setcounter{section}{0}\vspace{3\baselineskip}}
\newcommand{\readornot}{false}

\DeclareMathOperator{\dcup}{\dot{\cup}}

\begin{document}
\begin{center}{\LARGE \bf 幾何数理工学 ノート}\end{center}

\midashi{本講義の内容}
このノートは,2018年度幾何数理工学のまとめノートです。
\begin{enumerate}
    \item 位相空間:「近さ」が備わった空間概念。$\mathbb{R}^{n}$の一般化。
    \item 位相幾何:連続変形に対する不変性。
    \begin{itemize}
        \item 基本群
        \item ホモロジー
    \end{itemize}
    \item テンソル:座標変換に対する不変性。
    \begin{itemize}
        \item 3次元ベクトル${\displaystyle \begin{pmatrix}
            v_{1} & v_{2} & v_{3} 
        \end{pmatrix}^{\top}}$と${\displaystyle \begin{pmatrix}
            \pdif{f}{x_{1}} & \pdif{f}{x_{2}} & \pdif{f}{x_{3}}
        \end{pmatrix}^{\top}}$の違いは何か。
    \end{itemize}
\end{enumerate}

\midashi{参考書}
\begin{itemize}
    \item 内田伏一,「集合と位相」\footnote{amazonだと2,808円(税込み)。}
    \item Allen Hatcher,「Algebraic Topology」\footnote{\url{https://pi.math.cornell.edu/~hatcher/AT/ATpage.html}}
    \item 伊理正夫,韓太舜,「テンソル解析入門」\footnote{楽天だと2,097円(税込み)。}
\end{itemize}

\renewcommand{\baselinestretch}{0.1}
\tableofcontents
\renewcommand{\baselinestretch}{1.0}

\octopuspart{位相空間}
\expandafter\ifx\csname readornot\endcsname\relax
  \documentclass[uplatex]{jsarticle}
  \usepackage{octopus}
  \usepackage{url}
  \usepackage{tikz}

  \renewcommand{\proofname}{\textsf{証明}}
  \renewcommand{\postpartname}{章}
  \renewcommand{\thesection}{\thepart.\arabic{section}}
  \renewcommand{\thepart}{\arabic{part}}
  \makeatletter\renewcommand{\theequation}{\thesection.\arabic{equation}}\@addtoreset{equation}{section}\makeatother

  \newcommand{\octopuspart}[1]{\newpage\part{#1}\setcounter{section}{0}\vspace{3\baselineskip}}

  \DeclareMathOperator{\dcup}{\dot{\cup}}
  \begin{document}
\fi

\section{距離空間}

\begin{teigi}
    $X$を非空な集合,$d: \, X \times X \to \mathbb{R}$を実数値関数とする。次の三つの条件{\bf D1},{\bf D2},{\bf D3}を考える。
    
    \midashi{D1. } $\forall x, y \in X, \quad d(x,y) \ge 0$,\qquad $d(x,y) = 0 \quad \Longleftrightarrow \quad x = y$
    
    \midashi{D2. } $\forall x, y \in X, \quad d(x,y) = d(y,x)$
    
    \midashi{D3. } $\forall x, y \in X, \quad d(x,y) + d(y,z) \ge d(x,z)$
    
    $d$が{\bf D1},{\bf D2},{\bf D3}の条件を満たすとき,$d$を$X$上の\nw{距離関数}という。

    また,$X$あるいは$(X,d)$を\nw{距離空間}(metric space)という。
\end{teigi}

{\bf D2.}は対称性を表し,{\bf D3.}は三角不等式と呼ばれる。

{\bf D1.},{\bf D2.},{\bf D3.}を満たさないような関数として,例えば${d_{2}}^{2}$(2乗距離)がある。
これは三角不等式を満たさない。

\begin{rei}
    $X = \mathbb{R}^{n}$とする。
    \begin{align}
        d_{2} (x,y) &:= \sqrt{\sum_{i=1}^{n} (x_{i} - y_{i})^{2}}, \\
        d_{1} (x,y) &:= \sum_{i=1}^{n} |x_{i} - y_{i}|, \\
        d_{\infty} (x,y) &:= \max_{i} |x_{i} - y_{i}|
    \end{align}
    とすると,これらはどれも$X$上の距離関数である。なお,$(\mathbb{R}^{n}, d_{2})$を\nw{$n$次元Euclid空間}と呼ぶ。
\end{rei}
\begin{proof}
    ここでは$d_{2}$が{\bf D3.}の三角不等式を満たすことのみを示す。
    \begin{align}
        {d_{2}}^{2}(x,z)
        &= \sum_{i=1}^{n} (x_{i} - z_{i})^{2} = \sum_{i=1}^{n} (x_{i} - y_{i} + y_{i} - z_{i})^{2} \notag \\
        &= \sum_{i=1}^{n} \left( x_{i} - y_{i} \right)^{2} + \sum_{i=1}^{n} \left( y_{i} - z_{i} \right)^{2} + 2 \sum_{i=1}^{n} \left( x_{i} - y_{i} \right) \left( y_{i} - z_{i} \right) \notag \\
      &\le \sum_{i=1}^{n} \left( x_{i} - y_{i} \right)^{2} + \sum_{i=1}^{n} \left( y_{i} - z_{i} \right)^{2} + 2 \sqrt{\left( \sum_{i=1}^{n} \left( x_{i} - y_{i} \right)^{2} \right) \left( \sum_{i=1}^{n} \left( y_{i} - z_{i} \right)^{2} \right)} \label{eq:1.1:Cauchy}\\
        &= \left\{ \sqrt{\sum_{i=1}^{n} \left( x_{i} - y_{i} \right)^{2}} + \sqrt{\sum_{i=1}^{n} \left( y_{i} - z_{i} \right)^{2}} \right\}^{2}
         = \left( d_{2}(x,y) + d_{2} (y,z) \right)^{2}
    \end{align}
    である。ここで,式\eqref{eq:1.1:Cauchy}に至る変形にはCauchy-Schwarzの不等式が用いられている。
\end{proof}

\begin{rei}
    $X = \mathcal{C} [a,b]$を区間$[a,b] \subseteq \mathbb{R}$上の連続関数全体の集合とする。
    \begin{align}
        d_{2} (f,g) &:= \sqrt{ \int_{a}^{b} | f(t) - g(t) | ^{2} \dx{t}}, \\
        d_{\infty} (f,g) &:= \sup \sets{ \left| f(t) - g(t) \right| | t \in [a,b]}, \\
        d_{1} (f,g) &:= \int_{a}^{b} \left| f(t) - g(t) \right| \dx{t}
    \end{align}
    とすると,これらはどれも$X$上の距離関数である。
\end{rei}

\begin{proof}
    ここでは$d_{1}$が{\bf D3.}の三角不等式を満たすことと{\bf D1.}の零点が対角集合に限られることのみを示す。

    \midashi{{\bf D3.}について}
    \begin{align*}
        d_{1} (f,h)
        &= \int_{a}^{b} \left| f(t) - h(t) \right| \dx{t} \\
      &\le \int_{a}^{b} \left( \left| f(t) - g(t) \right| + \left| g(t) - h(t) \right| \right) \dx{t} \\
      &\le \int_{a}^{b} \left| f(t) - g(t) \right| \dx{t} +  \int_{a}^{b} \left| g(t) - h(t) \right| \dx{t} \\
        &= d_{1} (f,g) + d_{1} (g,h)
    \end{align*}
    で従う。

    \midashi{零点が対角集合に限られることについて}

    $f = g$ならば$d_{1} (f,g) = 0$は明らかであるので,問題はその逆である。

    もし,$f(x) \neq g(x)$ならばある$x_{0} \in [a,b]$で$\left| f(x_{0}) - g(x_{0}) \right|> 0$であり,
    $f-g$の連続性からある$\varepsilon > 0$が存在して,$y \in [x_{0} - \varepsilon, x_{0} + \varepsilon]$に対して
    $\left| f(y) - g(y) \right| > 0$である。
    したがって,
    \begin{equation*}
        \int_{a}^{b} \left| f(x) - g(x) \right| \dx{x} \ge \varepsilon \min_{x_{0} - \varepsilon \le y \le x_{0} + \varepsilon} \left| f(y) - g(y) \right| > 0
    \end{equation*}
    である。
\end{proof}

\begin{rei}
    $X = \sets{0,1}^{n}$とする。
    \begin{equation}
        d_{H}(x,y) = \# \sets{i | x_{i} \neq y_{i}}
    \end{equation}
    で定めると,これは$X$上の距離関数である。この距離関数は\nw{Hamming距離}と呼ばれる。
\end{rei}

\begin{rei}
    $G = (X,E)$を無向グラフとする。
    $d_{G}(x,y)$を$x$から$y$への最短路の長さとして定義すると,
    これは$G$上の距離関数である。
\end{rei}

以下,$(X,d)$を距離空間とする。

\begin{teigi}
    \midashi{1. } $a \in X$,$\varepsilon > 0$とする。集合$N(a,\varepsilon) := \sets{x \in X | d(a,x) < \varepsilon}$を,$a$の\nw{$\varepsilon$-近傍}という。

    \midashi{2. } $A \subseteq X$の\nw{内点}とは,次の条件を満たす点$x \in X$のこと:
    \begin{equation}
        \exists \varepsilon > 0, \quad N(x,\varepsilon) \subseteq A
    \end{equation}
    また,$A$の内点全体の集合$A^{\circ} = \sets{x \in X | N(x, \varepsilon) \subseteq A}$のことを$A$の\nw{内部}という。

    \midashi{3. } $A \subseteq X$の\nw{外点}とは,次の条件を満たす点$x \in X$のこと:
    \begin{equation}
        \exists \varepsilon > 0, \quad N(x , \varepsilon ) \cap A = \emptyset
    \end{equation}
    言い換えると,$A$の補集合の内点のこと。
    また,$A$の外点全体の集合$(X \setminus A)^{\circ}$を$A$の\nw{外部}という。

    \midashi{4. } $A \subseteq X$の\nw{境界点}とは,次の条件を満たす点$x \in X$のこと:
    \begin{equation}
        \forall \varepsilon > 0, \quad N(x, \varepsilon) \cap A \neq \emptyset, \, N(x,\varepsilon) \cap (X \setminus A) \neq \emptyset
    \end{equation}
    また,$A$の境界点全体の集合のことを$A$の\nw{境界}といい,記号$\partial A$で表す。

    \midashi{5.} $A \subseteq X$の\nw{触点}とは,次の条件を満たす点$x \in X$のこと:
    \begin{equation}
        \forall \varepsilon > 0, \quad N (x,\varepsilon) \cap A \neq \emptyset
    \end{equation}
    また,$A$の触点全体の集合のことを$A$の\nw{閉包}といい,記号$\overline{A}$で表す。
\end{teigi}

\begin{figure}[htbp]
    \centering
    \newcommand{\makeaxis}{
    \draw[thick, ->] (-2.5,0) -- (2.5,0) node [below] {$x$};%x軸
    \draw[thick, ->] (0,-2.5) -- (0,2.5) node [left] {$y$};%y軸
    \coordinate (O) at (0,0);
    \node at (O) [above left] {$a$};
    }
    \centering    
    \begin{tikzpicture}
        \makeaxis
        \draw[dashed] (O) circle (1.5);
        \draw[->] (O) -- node[midway, right]{$\varepsilon$} (45:1.5);
        \node at (-2,2) {$(\mathbb{R}^{2}, d_{2})$};
    \end{tikzpicture}
    \begin{tikzpicture}
        \makeaxis
        \draw[dashed] (1.5,0) -- (0,1.5) -- (-1.5,0) -- (0,-1.5) -- cycle;
        \node at (1.5,0) [below] {$a + \varepsilon$};
        \node at (-1.5,0) [below] {$a - \varepsilon$};
        \node at (-2,2) {$(\mathbb{R}^{2}, d_{1})$};
    \end{tikzpicture}
    \begin{tikzpicture}
        \makeaxis
        \draw[dashed] (-1.5,-1.5) -- (1.5,-1.5) -- (1.5,1.5) -- (-1.5,1.5) -- cycle;
        \node at (-1.5,0) [below left] {$a-\varepsilon$};
        \node at (0,-1.5) [below right] {$a-\varepsilon$};
        \node at (1.5,0) [above right] {$a+\varepsilon$};
        \node at (0,1.5) [above right] {$a+\varepsilon$};
        \node at (-2,2) {$(\mathbb{R}^{2}, d_{\infty})$};
    \end{tikzpicture}
    \caption{距離空間$(\mathbb{R}^{2},d)$上の近傍の違い}
    \label{fig:euclid_dist}
\end{figure}

\begin{figure}[htbp]
    \centering
    \begin{tikzpicture}
        \draw[dashed] (-30:2) arc (-30:210:2);
        \fill [black!20] (0,0) circle [x radius=1.97, y radius=1.97];
        \draw[thick] (210:2) arc (210:330:2);
        \draw[<-] (-50:2.1) -- (-50:2.4) node[right] {\bf 外点};
        \draw (20:0.5) node {\bf 内点};
        \draw[<-] (160:2.1) -- (160:2.6) node[left] {\bf 境界点};
        \draw (45:2.5) node {$A$};
    \end{tikzpicture}
    \caption{$A \subseteq X$の内部,外部,境界}
    \label{fig:in_ex_boundary}
\end{figure}

この定義から,内部,外部,境界点について
\begin{equation}
    X = A^{\circ} \dcup \partial A \dcup (X \setminus A)^{\circ}
\end{equation}
と互いに非交な集合に分解できる。
また,閉包について,次の二つが成り立つ。
\begin{align}
    & A ^ { \circ } \subseteq A \subseteq \overline { A } \\
    & \overline { A } = A ^ { \circ } \cup \partial A
\end{align}

\begin{teigi}
    \begin{equation*}
        \begin{array}{lcccl}
            A \subseteq X \text{:\nw{開集合}} & \defines & \forall x \in A, \quad \exists \varepsilon > 0, \quad  N(x , \varepsilon ) \subseteq A & \Longleftrightarrow & A = A^{\circ} \\
            A \subseteq X \text{:\nw{閉集合}} & \defines & 「\forall x \in X, \quad \forall \varepsilon > 0, \quad  N(x , \varepsilon ) \cap A \neq \emptyset \Longrightarrow x \in A」& \Longleftrightarrow & A = \overline{A} \\
        \end{array}
    \end{equation*}
\end{teigi}

\begin{hodai}
    $N(x,\varepsilon)$は開集合。特に,$N(x,\varepsilon) = N(x,\varepsilon)^{\circ}$。
\end{hodai}

\begin{proof}
    $y \in N ( x, \varepsilon)$をとる。$\delta := \varepsilon - d ( x , y ) > 0$とすると,
    $z \in N(y, \varepsilon)$に対して,三角不等式より
    \begin{equation}
        d(x,z) \le d(x,y) + d(y,z) < \varepsilon
    \end{equation}
である。よって,$N(y,\delta) \subseteq N(x,\varepsilon)$となる。
\end{proof}

\begin{prop}
    $\left( A^{\circ} \right)^{\circ} = A^{\circ}$,$\overline{\overline{A}} = \overline{A}$
\end{prop}

\begin{proof}
    $\left( A^{\circ} \right)^{\circ} \subseteq A^{\circ}$より,$\supseteq$向きを示す。
    $x \in A^{\circ}$とする。
    \begin{equation}
       \exists \varepsilon > 0, \quad N(x, \varepsilon) \subseteq A
    \end{equation}
    である。$N(x, \varepsilon) = N(x, \varepsilon)^{\circ} \subseteq A^{\circ}$,
    つまり$N(x, \varepsilon) \subseteq A^{\circ}$である。
    $x \in \left( A^{\circ} \right)^{\circ}$である。

    $\overline{\overline{A}} \subseteq \overline{A}$より,$\subseteq$向きを示す。
    $x \in \overline{\overline{A}}$とする。
    \begin{equation}
        \forall \varepsilon > 0, \quad N(x,\varepsilon) \cap \overline{A} \neq \emptyset
    \end{equation}
    である。$y \in N(x, \varepsilon) \cap \overline{A}$,$\delta := \varepsilon - d(x,y) > 0$とする。
    $y \in \overline{A}$より$N(y, \delta) \cap A \neq \emptyset$である。
    $N(y, \delta) \subseteq N(x,\varepsilon)$より,\footnote{上と同じロジックで}\,
    $\emptyset \neq N(y, \delta) \cap A \subseteq N(x, \varepsilon) \cap A$である。よって,$x \in \overline{A}$である。
\end{proof}

\begin{prop}
    \midashi{1. } $A$:開 $\Longrightarrow$ $X \setminus A$:閉

    \midashi{2. } $A$:閉 $\Longrightarrow$ $X \setminus A$:開
\end{prop}

\begin{proof}
    \begin{align*}
        A\text{が開でない} & \Longleftrightarrow \exists x \in A, \quad \forall \varepsilon > 0, \quad N(x, \varepsilon) \not\subseteq A \\
        & \Longleftrightarrow \exists x \in A, \quad \forall \varepsilon > 0, \quad N(x, \varepsilon) \cap (X \setminus A) \neq \emptyset \\
        & \Longleftrightarrow X \setminus A \text{が閉でない}
    \end{align*}
    ここで2行目の同値性は,$x$は$X \setminus A$の触点であるが$X \setminus A$でないことから従う。
\end{proof}

開集合全体の集合を\nw{開集合系}といい,記号$\mathcal{O}$で表す。
閉集合全体の集合を\nw{閉集合系}といい,記号$\mathfrak{A}$で表す。

\begin{prop}
    \label{prop:open_set_axiom}
    開集合系$\mathcal{O}$について以下が成り立つ。

    \midashi{1. } $X \in \mathcal{O}$,$\emptyset \in \mathcal{O}$

    \midashi{2. } $O_{1}, O_{2}, \cdots, O_{k} \in \mathcal{O} \Longrightarrow O_{1} \cap O_{2} \cap \cdots \cap O_{k} \in \mathcal{O}$

    \midashi{3. } 任意の集合$\Lambda$に対して$O_{\lambda} \in \mathcal{O}$($\forall \lambda \in \Lambda$) ${\displaystyle \Longrightarrow \bigcup_{\lambda \in \Lambda} O_{\lambda} \in \mathcal{O}}$
\end{prop}

\begin{proof}
    \midashi{1. } obvious.

    \midashi{2. } $x \in O_{1} \cap O_{2} \cap \dots \cap O_{k}$をとる。$i = 1,2,\dots,k$に対して$\varepsilon_{i} > 0$が存在して
    \begin{equation}
        N(x_{i}, \varepsilon_{i}) \subseteq O_{i}
    \end{equation}
    である。${\displaystyle \varepsilon := \min_{i} \varepsilon_{i}}$とすると,
    $N(x, \varepsilon) \subseteq O_{1} \cap O_{2} \cap \dots \cap O_{k}$である。
    $x$は$O_{1} \cap O_{2} \cap \dots \cap O_{k}$の内点である。よって,$\left( O_{1} \cap O_{2} \cap \dots \cap O_{k} \right)^{\circ} = O_{1} \cap O_{2} \cap \dots \cap O_{k}$である。

    \midashi{3. } ${x \in \displaystyle \bigcup_{\lambda \in \Lambda} O_{\lambda}}$とすると,
    $\exists \lambda \in \Lambda, \quad x \in O_{\lambda}$である。${\displaystyle \exists \varepsilon > 0, \quad N(x, \varepsilon) \subseteq O_{\lambda} \subseteq \bigcup_{\lambda} O_{\lambda}}$とする。
\end{proof}

\begin{prop}
    閉集合系$\mathfrak{A}$について以下が成り立つ。
    
    \midashi{1. } $X \in \mathfrak{A}$,$\emptyset \in \mathfrak{A}$

    \midashi{2. } $A_{1}, A_{2}, \cdots, A_{k} \in \mathfrak{A} \Longrightarrow A_{1} \cup A_{2} \cup \cdots \cup A_{k} \in \mathfrak{A}$

    \midashi{3. } 任意の集合$\Lambda$に対して$A_{\lambda} \in \mathfrak{A}$($\forall \lambda \in \Lambda$) ${\displaystyle \Longrightarrow \bigcap_{\lambda \in \Lambda} A_{\lambda} \in \mathfrak{A}}$
\end{prop}

\begin{proof}
    \midashi{1. } $\mathfrak{A} = \sets{ X \setminus O | O \in \mathcal{O}}$に注意すると明らか。

    \midashi{2. } $A_{i} = X \setminus O_{i}$とかくと,
    \begin{equation}
        A_{1} \cup A_{2} \cup \dots \cup A_{k}
        = (X \setminus O_{1}) \cup (X \setminus O_{2}) \cup \dots \cup (X \setminus O_{k})
        = X \setminus (O_{1} \cap O_{2} \cap \dots \cap O_{k}) \in \mathfrak{A}
    \end{equation}
    である。

    \midashi{3. } ${\displaystyle \bigcap_{\lambda \in \Lambda} A_{\lambda} = \bigcap_{\lambda \in \Lambda} (X \setminus O_{\lambda}) = X \setminus \bigcup_{\lambda \in \Lambda} O_{\lambda}}$であるから従う。
\end{proof}

収束性,連続性なども距離空間で定義できる。

\begin{teigi}
    \midashi{1. } $a_{i} \in X$($i=1,2,\dots$):\nw{収束}とは,
    \begin{equation}
        \lim_{i \to \infty} a_{i} = a \defines \lim_{i \to \infty} d(a, a_{i}) = 0
    \end{equation}
    
    \midashi{2. } $f:X \longrightarrow \mathbb{R}$が$a \in X$で\nw{連続}とは,
    \begin{equation}
        \forall \varepsilon > 0, \quad \exists \delta > 0, d(a,x) < \delta \Longrightarrow \left| f(a) - f(x) \right| < \varepsilon
    \end{equation}
\end{teigi}

では,$X$上の異なる距離$d$,$d'$の下で連続性や収束性も変わるのか。

$\longrightarrow$位相空間の考え方へ

実際,開集合系を与える(空間に「位相」を与える)ことによって,
連続性や収束性が定義でき,
開集合系が同じとき,連続性や収束性も等しくなる。

\begin{rei}
    $(\mathbb{R}^{n},d_{2})$と$(\mathbb{R}^{n},d_{\infty})$が定義する開集合系は等しい。(位相は等しい)
    \begin{proof}
        $A$:$d_{2}$の下で開集合とする。
        $x \in A$をとる。$\exists \varepsilon > 0, \quad N_{2}(x,\varepsilon) \subseteq A$である。
        $\exists c > 0, \quad N_{\infty} (x, c \varepsilon) \subseteq N_{2} (x, \varepsilon) \subseteq A$である。
        $A$は$d_{\infty}$の下でも開。逆も同様に言える。 
    \end{proof}
    結局,連続性や収束性はどちらの距離で考えても同じになる。
\end{rei}

\begin{rei}
    $\mathbb{R}^{n}$上のノルム$\left\| \cdot \right\|$とは,
    \begin{itemize}
        \vspace{-0.5\baselineskip}
        \item $\left\| v \right\| \ge 0$,\qquad $\left\| v \right\| = 0 \Longleftrightarrow v = 0$
        \item $\forall a \in \mathbb{R}$,$v \in \mathbb{R}^{n}$,$\left\| av \right\| = \left| a \right| \left\| v \right\|$
        \item $\forall u,v \in \mathbb{R}^{n}$,$\left\| u + v \right\| \le \left\| u \right\| + \left\| v \right\|$
        \vspace{-0.5\baselineskip}
    \end{itemize}
    を満たすものである。このとき,$d(x,y) := \left\| x - y \right\|$で定義すると距離関数である。
\end{rei}

\midashi{知っておいてほしいこと}
\begin{itemize}
    \vspace{-0.5\baselineskip}
    \item $\mathbb{R}^{n}$上のどんなノルムを定める「位相」は等しい。
    \item 無限次元では,そうはいかない({\bf らしい})。
\end{itemize}

\expandafter\ifx\csname readornot\endcsname\relax
  \end{document}
\fi % 距離空間
\expandafter\ifx\csname readornot\endcsname\relax
  \documentclass[uplatex]{jsarticle}
  \usepackage{octopus}
  \usepackage{url}

  \renewcommand{\proofname}{\textsf{証明}}
  \renewcommand{\postpartname}{章}
  \renewcommand{\thesection}{\thepart.\arabic{section}}
  \renewcommand{\thepart}{\arabic{part}}
  \makeatletter\renewcommand{\theequation}{\thesection.\arabic{equation}}\@addtoreset{equation}{section}\makeatother
  
  \newcommand{\octopuspart}[1]{\newpage\part{#1}\setcounter{section}{0}\vspace{3\baselineskip}}
  
  \DeclareMathOperator{\dcup}{\dot{\cup}}
  \begin{document}
  \fi
\renewcommand{\thesubsection}{\thepart.\arabic{section}.\arabic{subsection}}
  
\section{位相空間}

$X$を非空な集合とする。$X$は「\nw{位相}」(topology)を入れて「空間」にする。
「位相」を入れるとは,$X$の開集合族を指定することである。

\begin{teigi}[位相]
  以下を満たす開集合系$\mathcal{O} \subseteq 2^{X}$を\nw{位相}という。

  \midashi{O1. } $X \in \mathcal{O}$,$\emptyset \in \mathcal{O}$

  \midashi{O2. } $O_{1}, O_{2}, \cdots, O_{k} \in \mathcal{O} \Longrightarrow O_{1} \cap O_{2} \cap \cdots \cap O_{k} \in \mathcal{O}$

  \midashi{O3. } 任意の集合$\Lambda$に対して$O_{\lambda} \in \mathcal{O}$($\forall \lambda \in \Lambda$) ${\displaystyle \Longrightarrow \bigcup_{\lambda \in \Lambda} O_{\lambda} \in \mathcal{O}}$

  $\mathcal{O}$の元を\nw{開集合}という。また,開集合族$\mathcal{O}$が指定された集合$X$を\nw{位相空間}という。
  これを$(X,\mathcal{O})$で表すこともある。
\end{teigi}

\begin{rei}
  \begin{itemize}
    \item \nw{離散位相} $\mathcal{O} = 2^{X}$
    \item \nw{密着位相} $\mathcal{O} = \sets{\emptyset, X}$
    \item 距離空間$(X,d)$;$\mathcal{O} := \sets{O \subseteq X | \forall x \in O, \quad \exists \varepsilon > 0, \quad N(x, \varepsilon) \subseteq O}$
    (これは距離空間のところで定義した開集合の集まり)
  \end{itemize}
\end{rei}

距離空間のときに定義したいくつかの用語を位相空間の言葉で記述し直す。

\begin{itemize}
  \vspace{-0.5\baselineskip}
  \item $A \subseteq X$の内部$A^{\circ}$
  \begin{equation}
    A^{\circ} := \bigcup \sets{O \in \mathcal{O} | O \subseteq A} \in \mathcal{O}
  \end{equation}
  これは$A$に含まれる最大の開集合のこと。$A$の内点$x$とは$A^{\circ}$の元のこと。
  \item 閉集合とは,ある開集合の補集合になっているものとする。
  すると,閉集合族$\mathfrak{A}$は以下を満たす。:

  \midashi{A1. } $X \in \mathfrak{A}$,$\emptyset \in \mathfrak{A}$

  \midashi{A2. } $F_{1}, F_{2}, \dots, F_{k} \in \mathfrak{A} \Longrightarrow F_{1} \cup F_{2} \cup \dots \cup F_{k} \in \mathfrak{A}$

  \midashi{A3. } 任意の集合$\Lambda$に対して$F_{\lambda} \in \mathfrak{A} $($\lambda \in \Lambda$)$\Longrightarrow {\displaystyle \bigcap_{\lambda} F_{\lambda} \in \mathfrak{A}}$

  \item $A$の閉包$\overline{A}$
  \begin{equation}
    \overline{A} := \bigcap \sets{F \in \mathfrak{A} | A \subseteq F} \in \mathfrak{A}
  \end{equation}
  これは$A$を含む最小の閉集合のこと。$A$の触点$x$とは$\overline{A}$の元のこと。
  \item $N$が$x \in X$の近傍であるとは,$x \in N^{\circ}$となること,すなわち,
  \begin{equation}
    \exists O \in \mathcal{O}, \quad x \in O \subseteq N
  \end{equation}
  となること。特に$N$が開集合であるとき,$N$を$x$の開近傍という。
  \item $\mathcal{N}(x)$:$x$の近傍全体
  \vspace{-0.5\baselineskip}
\end{itemize}

位相の与え方にはいろいろある。

\begin{enumerate}
  \item {\bf (A1)},{\bf (A2)},{\bf (A3)}を満たす集合族$\mathfrak{A}$(閉集合族)を指定する。
  \item 各$A \subseteq X$に$A^{\circ}$を対応させる写像$2^{X} \longrightarrow 2^{X}$を指定する。(\nw{開核作用子})
  \item 各$A \subseteq X$に$\overline{A}$を対応させる写像$2^{X} \longrightarrow 2^{X}$を指定する。(\nw{閉包作用子})
  \item 各点$x$に$\mathcal{N}(x)$を対応させる写像$X \longrightarrow 2^{2^{X}}$を指定する。
\end{enumerate}

ただし,指定する写像はどんなものでもいいわけではない。
それぞれある条件を満たすような写像に限られる。
例えば,閉包作用子$\tau \colon 2^{X} \longrightarrow 2^{X}$は次を満たす必要がある。
\begin{itemize}
  \vspace{-0.5\baselineskip}
  \item $\tau (\emptyset) = \emptyset$
  \item $A \subseteq \tau(A)$
  \item $\tau (A \cup B) = \tau (A) \cup \tau (B)$
  \item $\tau (\tau (A)) = \tau (A)$
  \vspace{-0.5\baselineskip}
\end{itemize}

これらから自然に開集合族が決まる。
例えば,閉包作用子に対しては,$A = \tau (A)$となるような$A$を閉集合と定める。
詳細は演習。

\sukima \midashi{\large 連続写像}

\renewcommand{\arraystretch}{0.8}
\vspace{0.2\baselineskip}
\begin{tabular}{ll@{\,}l@{\,}l}
  位相空間 & $X$, & $\mathcal{O}_{X}$:開集合族, & $\mathfrak{A}_{X}$:閉集合族 \\
           & $Y$, & $\mathcal{O}_{Y}$:開集合族, & $\mathfrak{A}_{Y}$:閉集合族
\end{tabular}
\renewcommand{\arraystretch}{1}

\begin{teigi}[連続性]
  写像$f \colon X \longrightarrow Y$が$x \in X$で\nw{連続} $\defines$ 
  「$\forall N$:$f(x)$の近傍 $\Longrightarrow$ $f^{-1}(N)$は$x$の近傍」
\end{teigi}

\midashi{距離空間の連続性との同値性}
\begin{align*}
  x \in X \text{で連続} & \Longleftrightarrow \forall \varepsilon > 0, \quad \exists \delta > 0, \quad d_{X} (x,y) < \delta \Longrightarrow d_{Y}(f(x), f(y)) < \varepsilon \\
  & \Longleftrightarrow \forall \varepsilon > 0, \quad \exists \delta > 0, \quad N(x,\delta) \subseteq f^{-1} (N(f(x), \varepsilon)) \\
  & \Longleftrightarrow f(x) \text{の任意の近傍の逆像は} x \text{の近傍}
\end{align*}

\begin{teiri}
  以下の{\bf (1)},{\bf (2)},{\bf (3)}は同値。:
  
  \midashi{(1)} $f$は$X$の各点で連続

  \midashi{(2)} $\forall O \in \mathcal{O}_{Y}, \quad f^{-1}(O) \in \mathcal{O}_{X}$

  \midashi{(3)} $\forall F \in \mathfrak{A}_{Y}, \quad f^{-1}(F) \in \mathfrak{A}_{X}$
\end{teiri}

\begin{proof}
  \midashi{(1) $\Longrightarrow$ (2):} $O \in \mathcal{O}_{Y}$を任意にとり,$x \in f^{-1}(O)$とする。$O$は$f(x)$の開近傍である。
  連続性の定義からある$x$の開近傍$O_{x}$が存在して,$O_{x} \subseteq f^{-1} (O)$である。
  \begin{equation}
    f^{-1} (O) = \bigcup_{x \in f^{-1}(O)} \sets{x} \subseteq \bigcup_{x \in f^{-1}(O)} O_{x} \subseteq f^{-1}(O)
  \end{equation}
  であるから,${\displaystyle f^{-1}(O) = \bigcup_{x \in f^{-1} (O)} O_{x} \in \mathcal{O}_{X}}$である。

  \midashi{(2) $\Longrightarrow$ (3):} $F \in \mathfrak{A}_{Y}$を任意にとる。$Y \setminus F \in \mathcal{O}_{Y}$であり,
  したがって,$f^{-1} (Y \setminus F) = X \setminus f^{-1}(F) \in \mathcal{O}_{X}$である。
  よって,$f^{-1} (F) \in \mathfrak{A}_{X}$である。

  \midashi{(3) $\Longrightarrow$ (2):} $O \in \mathcal{O}_{Y}$を任意にとる。$Y \setminus O \in \mathfrak{A}_{Y}$であり,
  したがって,$f^{-1} (Y \setminus O) = X \setminus f^{-1} (O) \in \mathfrak{A}_{X}$である。
  よって,$f^{-1} (O) \in \mathcal{O}_{X}$である。

  \midashi{(2) $\Longrightarrow$ (1):} $x \in X$を任意にとる。$N$を$f(x)$の近傍として,その内部を$O := N^{\circ}$とすれば$f(x) \in O$である。
  {\bf (2)}より$f^{-1}(O) \in \mathcal{O}_{X}$であってこれは$x$の開近傍である。
  $x \in f^{-1} (O) \subseteq f^{-1} (N)$である。
\end{proof}

\begin{teigi}[同相]
  \midashi{(1)} $f \colon X \longrightarrow Y$が\nw{連続} $\defines$ 上の{\bf (1)}から{\bf (3)}のどれかの条件を満たす
  
  \midashi{(2)} $f \colon X \longrightarrow Y$が\nw{同相}(homeomorphic) $\defines$ $f$:連続全単射 かつ $f^{-1} \colon Y \longrightarrow X$:連続
  
  $X$と$Y$の間に同相写像が存在するとき,$X$と$Y$は\nw{同相}あるいは\nw{位相同型}といい,
  記号$X \simeq Y$で表す。
\end{teigi}

\midashi{演習:}全単射連続写像であって同相でないものの例を与えよ。

\begin{teigi}[位相の強弱]
  $X$上の2つの位相$\mathcal{O}$,$\mathcal{O}'$を考える。
  $\mathcal{O} \subseteq \mathcal{O}'$のとき,
  「$\mathcal{O}$は$\mathcal{O}'$より\nw{弱い}位相である」,「$\mathcal{O}'$は$\mathcal{O}$より\nw{強い}位相である」という。
\end{teigi}

つまり,$\mathcal{O}$で連続な写像は,$\mathcal{O}'$でも連続な写像になる。
{\footnotesize「使ったことない」「弱い強い混乱するわロクでもない概念」などと。}

\sukima \midashi{\large いろいろな位相}

\newcommand{\dtimes}{\times\mkern-16mu\times}

\begin{teigi}[誘導位相]
  $X$:集合,$(Y,\mathcal{O}_{Y})$:位相空間,$f \colon X \longrightarrow Y$とする。
  \begin{equation}
    \mathcal{O}_{X} := \sets{f^{-1}(O) | O \in \mathcal{O}_{Y}}
  \end{equation}
  を$f$による\nw{誘導位相}という。
\end{teigi}

$f^{-1} (O \cup O') = f^{-1} (O) \cup f^{-1} (O')$,
$f^{-1} (O \cap O') = f^{-1} (O) \cap f^{-1} (O')$に注意すると,位相の定義の条件を満たすことは理解できる。

\begin{teigi}[相対位相]
  $(X, \mathcal{O}_{X})$:位相空間,$Z \subseteq X$:部分集合。
  \begin{equation}
    \mathcal{O}_{Z} := \sets{O \cap Z | O \in \mathcal{O}_{X}}
  \end{equation}
  とすると,$(Z, \mathcal{O}_{Z})$は$(X, \mathcal{O}_{X}$の部分(位相)空間をなす。
  この位相を\nw{相対位相}という。
\end{teigi}

相対位相は包含写像$Z \hookrightarrow X$による誘導位相でもある。

\sukima \midashi{直積位相}

$(X, \mathcal{O}_{X})$,$(Y, \mathcal{O}_{Y})$:位相空間として,
直積$X \times Y = \sets{(x,y) | x \in X, y \in Y}$に位相を与えたい。しかし,
\begin{equation}
  \mathcal{B} := \sets{O \times O' | O \in \mathcal{O}_{X}, O' \in \mathcal{O}_{Y}}
\end{equation}
は開集合の条件を満たさない。そこで$\mathcal{B}$が「生成する」位相$\mathcal{O}_{X} \dtimes \mathcal{O}_{Y}$を
\begin{equation}
  \mathcal{O}_{X} \dtimes \mathcal{O}_{Y} := \sets{ \bigcup_{W \in \mathcal{B}'} W | \mathcal{B}' \subseteq \mathcal{B}}
\end{equation}
で定めると,$(X \times Y, \mathcal{O}_{X} \dtimes \mathcal{O}_{Y})$は位相空間になる。

3つ以上の位相空間に対しても同様にして直積位相を定めることができる。

\begin{rei}[$(\mathbb{R}^{2}, d_{2})$の位相]
  $x$の$d_{2}$の下での$\varepsilon$-近傍に含まれる$d_{\infty}$の下での$\varepsilon'$-近傍をとることができる。
  $\mathbb{R}^{2}$の開集合は$d_{\infty}$の近傍たちの和集合で書くことができる。
  $d_{\infty}$の$\varepsilon'$-近傍はすべて$\mathcal{B}$の元である。
  よって,$\mathbb{R}^{2}$の位相は$\mathbb{R} \times \mathbb{R}$の直積位相に等しい。
\end{rei}

\sukima \midashi{商位相}

2つの位相空間を貼り合わせて新たな位相空間を作りたい。
% 貼り合わせる図 if needed

\begin{itemize}
  \vspace{-0.5\baselineskip}
  \item まず$X$と$Y$の直和$X \amalg Y$を考える。$X \amalg Y$に位相を入れる。
  $\mathcal{O}_{X \amalg Y} := \sets{O \cup O' | O \in \mathcal{O}_{X}, O' \in \mathcal{O}_{Y}}$
  \item 同一視したい点たちを同値関係$\sim$で同一視して,商集合$(X \amalg Y) / \sim$を作る。
  \item 商集合に商位相を入れる。
  \vspace{-0.5\baselineskip}
\end{itemize}

位相空間$(X,\mathcal{O})$,$\sim$:$X$上の同値関係,
$X / \sim$:商集合,$\varphi \colon X \longrightarrow X / \sim$:自然な射影(自身を代表元とする同値類を返す)
\begin{equation}
  {\mathcal{O} / \sim} := \sets{H \subseteq {X / \sim} | \varphi^{-1}(H) \subseteq \mathcal{O}}
\end{equation}
$({X / \sim}, {\mathcal{O} / \sim})$は位相空間になる。
%よくわからない絵

\sukima \midashi{\large 重要な位相空間の例}

$\mathbb{R}^{n}$:Euclid空間(位相はEuclid距離から)

\begin{itemize}
  \vspace{-0.5\baselineskip}
  \item $S^{n}$:$n$次元球面
  \begin{equation}
    S^{n} := \sets{x \in \mathbb{R}^{n+1} | \sum_{j=1}^{n+1} {x_{j}}^{2} = 1}
  \end{equation}
  位相は$\mathbb{R}^{n+1}$からの相対位相とする。

  \item $D^{n}$:$n$次元ディスク
  \begin{equation}
    D^{n} := \sets{x \in \mathbb{R}^{n} | \sum_{j=1}^{n} {x_{j}}^{2} \le 1}
  \end{equation}

  \begin{center}
    $D^{2} \simeq$
    \begin{tikzpicture}[baseline=12pt] \filldraw[fill opacity=.3,draw = black] (0,0) rectangle (1,1); \end{tikzpicture}
      , \quad $S^{2} \simeq {D^{2} / \sim} \simeq$
    \begin{tikzpicture}[baseline=12pt]
      \filldraw[fill opacity=.3, draw = black] (0,0) rectangle (1,1);
      \draw (0.55,-0.1) -- (0.45, 0) -- (0.55,0.1);
      \draw (0.9,0.45) -- (1, 0.55) -- (1.1,0.45);
      \draw (-0.1,0.6) -- (0, 0.5) -- (0.1,0.6);
      \draw (-0.1,0.5) -- (0, 0.4) -- (0.1,0.5);
      \draw (0.4,1.1) -- (0.5,1) -- (0.4,0.9);
      \draw (0.5,1.1) -- (0.6,1) -- (0.5,0.9);
    \end{tikzpicture}
  \end{center}

  \item $P^{n}$:射影空間
  \begin{equation}
    P^{n} := (\mathbb{R}^{n+1} \setminus \sets{0}) / \sim
  \end{equation}
  このときの同値関係は
  \begin{equation}
    (x_{1},x_{2},\dots,x_{n+1}) \sim (y_{1},y_{2},\dots,y_{n+1}) \defines \exists \alpha \in \mathbb{R}, \quad (x_{1},x_{2},\dots,x_{n+1}) = \alpha (y_{1},y_{2},\dots,y_{n+1})
  \end{equation}
  で定めている。
  位相は$\mathbb{R}^{n+1}$の部分空間$+$商位相で定める。
  ちなみに$S_{+}^{n}$は上半球面として
  \begin{align}
    P^{n} &\simeq S^{n} / (\text{対蹠点を同一視}) \\
    &\simeq S_{+}^{n} / (\text{対蹠点を同一視}) \\
    &\simeq D^{n-1} / (\text{対蹠点を同一視})
  \end{align}
  が成り立つ。
  %図

  \item $T^{n}$:$n$次元トーラス
  \begin{equation}
    T^{n} := \underbrace{S^{1} \times S^{1} \times \dots \times S^{1}}_{n}
  \end{equation}
  位相は直積位相で定める。

  \item M\"obiusの輪

  \begin{tikzpicture}[baseline=12pt]
    \filldraw[fill opacity=.3, draw = black] (0,0) rectangle (2,1);
    \draw (1.9,0.55) -- (2, 0.45) -- (2.1,0.55);
    \draw (-0.1,0.45) -- (0, 0.55) -- (0.1,0.45);
  \end{tikzpicture}

  \item Kleinの壺

  \begin{tikzpicture}[baseline=12pt]
    \filldraw[fill opacity=.3, draw = black] (0,0) rectangle (2,1);
    \draw (1.9,0.55) -- (2, 0.45) -- (2.1,0.55);
    \draw (-0.1,0.55) -- (0, 0.45) -- (0.1,0.55);
    \draw (0.9,1.1) -- (1,1) -- (0.9,0.9);
    \draw (1,1.1) -- (1.1,1) -- (1,0.9);
    \draw (1,0.1) -- (0.9,0) -- (1,-0.1);
    \draw (1.1,0.1) -- (1,0) -- (1.1,-0.1);
  \end{tikzpicture}
  \vspace{-0.5\baselineskip}
\end{itemize}
% 適宜図を挿入

\sukima
\begin{teigi}(Hausdorff空間)
  $X$:位相空間。
  \vspace{-0.5\baselineskip}
  \begin{equation*}
    \begin{array}{lll}
      X \colon \text{\nw{Hausdorff空間}} & \defines & 「\forall x,y \in X, \quad x \neq y \Longrightarrow \exists U \in \mathcal{O}, \quad
      \exists V \in \mathcal{O}, \quad x \in U, \quad y \in V, \quad U \cap V = \emptyset」
    \end{array}
  \end{equation*}
\end{teigi}

\begin{prop}
  距離空間はHausdorff空間である。
\end{prop}

\begin{teigi}[多様体]
  $M$:Hausdorff空間。
  \vspace{-0.5\baselineskip}
  \begin{equation*}
    \begin{array}{lll}
      M \colon \text{\nw{$n$次元多様体}} & \defines & \forall p \in M, \quad \exists U \colon \text{$p$の開近傍}, \quad
      \exists U' \colon \mathbb{R}^{n} \text{の開集合}, \quad
      \exists \varphi \colon U \longrightarrow U' \subseteq \mathbb{R}^{n}, \quad \varphi \colon \text{同相写像}
    \end{array}
  \end{equation*}
\end{teigi}

\begin{rei}
  $S^{n}$,$P^{n}$,$T^{n}$は多様体。
\end{rei}

\midashi{演習:}$f \colon \mathbb{R}^{n} \longrightarrow \mathbb{R}$に対して
$\sets{x \in \mathbb{R}^{n} | f(x) = 0}$が多様体となるのはどんなときか。

\begin{rei}[単体複体]
  $S_{\lambda}$($\lambda \in \Lambda$):単体(これも位相空間の一種)の集合
  \begin{equation}
    X := \coprod_{\lambda \in \Lambda} S_{\lambda} = \bigcup_{\lambda \in \Lambda} S_{\lambda} \times \sets{\lambda}
  \end{equation}
  として,ここに同値関係$\sim$を貼り合わせの同一視の関係として定める。
  商空間$K = X / \sim$で全体の単体複体を表す。
\end{rei}

\expandafter\ifx\csname readornot\endcsname\relax
  \end{document}
\fi
 % 位相空間
\expandafter\ifx\csname readornot\endcsname\relax
  \documentclass[uplatex]{jsarticle}
  \usepackage{octopus}
  \usepackage{url}

  \renewcommand{\proofname}{\textsf{証明}}
  \renewcommand{\postpartname}{章}
  \renewcommand{\thesection}{\thepart.\arabic{section}}
  \renewcommand{\thepart}{\arabic{part}}
  \makeatletter\renewcommand{\theequation}{\thesection.\arabic{equation}}\@addtoreset{equation}{section}\makeatother

  \newcommand{\octopuspart}[1]{\newpage\part{#1}\setcounter{section}{0}\vspace{3\baselineskip}}

  \DeclareMathOperator{\dcup}{\dot{\cup}}
  \begin{document}
\fi

\section{連結性}

\expandafter\ifx\csname readornot\endcsname\relax
  \end{document}
\fi % 連結性
\expandafter\ifx\csname readornot\endcsname\relax
  \documentclass[uplatex]{jsarticle}
  \usepackage{octopus}
  \usepackage{url}

  \renewcommand{\proofname}{\textsf{証明}}
  \renewcommand{\postpartname}{章}
  \renewcommand{\thesection}{\thepart.\arabic{section}}
  \renewcommand{\thepart}{\arabic{part}}
  \makeatletter\renewcommand{\theequation}{\thesection.\arabic{equation}}\@addtoreset{equation}{section}\makeatother

  \newcommand{\octopuspart}[1]{\newpage\part{#1}\setcounter{section}{0}\vspace{3\baselineskip}}

  \DeclareMathOperator{\dcup}{\dot{\cup}}
  \begin{document}
\fi

\section{コンパクト性}
$(X, \mathcal{O})$:位相空間,$A \subseteq X$とする。

\begin{teigi}[被覆]
  $\mathcal{C} \subseteq 2^{X}$:$A$の\nw{被覆} $\defines$ ${\displaystyle A \subseteq \bigcup_{C \in \mathcal{C}} C}$

  特に,$\mathcal{C} \subseteq \mathcal{O}$のとき,$\mathcal{C}$を\nw{開被覆}という。
\end{teigi}

\begin{teigi}[コンパクト]
  \midashi{(1)} $A \subseteq X$:\nw{コンパクト} $\defines$ $\forall \mathcal{C}$:$A$の開被覆,$\exists O_{1}, \dots, O_{k} \in \mathcal{C}$,${\displaystyle A \subseteq \bigcup_{j=1}^{k} O_{j}}$

  \midashi{(2)} $X$がコンパクトのとき,$(X, \mathcal{O})$をコンパクト空間という。
\end{teigi}

$A \subseteq X$がコンパクトであることは,標語的に「任意の開被覆は有限部分開被覆を含む」ということができる。
また,$A \subseteq X$がコンパクトであることは,$(A, \mathcal{O}_{A})$がコンパクト空間であることと同値である。
ただし,$\mathcal{O}_{A}$は相対位相。

\begin{hodai}
  $A_{1}, \dots, A_{k} \subseteq X$ :コンパクト $\Longrightarrow$ $A_{1} \cup \dots \cup A_{k}$:コンパクト
\end{hodai}

\begin{hodai}
  $(X, \mathcal{O})$:コンパクト $\Longrightarrow$ $A \in \mathfrak{A}$:コンパクト
\end{hodai}

\begin{proof}
  $A$の開被覆と$X \setminus A$が$X$の開被覆になっていることから従う。
\end{proof}

\expandafter\ifx\csname readornot\endcsname\relax
  \end{document}
\fi % コンパクト性

\octopuspart{位相幾何}
連続変形に対する不変性

\expandafter\ifx\csname readornot\endcsname\relax
  \documentclass[uplatex]{jsarticle}
  \usepackage{octopus}
  \usepackage{url}

  \renewcommand{\proofname}{\textsf{証明}}
  \renewcommand{\postpartname}{章}
  \renewcommand{\thesection}{\thepart.\arabic{section}}
  \renewcommand{\thepart}{\arabic{part}}
  \makeatletter\renewcommand{\theequation}{\thesection.\arabic{equation}}\@addtoreset{equation}{section}\makeatother

  \newcommand{\octopuspart}[1]{\newpage\part{#1}\setcounter{section}{0}\vspace{3\baselineskip}}

  \DeclareMathOperator{\dcup}{\dot{\cup}}
  \begin{document}
\fi

\section{ホモトピー}

\expandafter\ifx\csname readornot\endcsname\relax
  \end{document}
\fi % ホモトピー
\expandafter\ifx\csname readornot\endcsname\relax
  \documentclass[uplatex]{jsarticle}
  \usepackage{octopus}
  \usepackage{url}

  \renewcommand{\proofname}{\textsf{証明}}
  \renewcommand{\postpartname}{章}
  \renewcommand{\thesection}{\thepart.\arabic{section}}
  \renewcommand{\thepart}{\arabic{part}}
  \makeatletter\renewcommand{\theequation}{\thesection.\arabic{equation}}\@addtoreset{equation}{section}\makeatother

  \newcommand{\octopuspart}[1]{\newpage\part{#1}\setcounter{section}{0}\vspace{3\baselineskip}}

  \DeclareMathOperator{\dcup}{\dot{\cup}}
  \begin{document}
\fi

\section{基本群}

\midashi{これからの展望}

\begin{center}
  \begin{tikzpicture}
    \draw (0,0) circle [x radius=1, y radius=0.5] node {空間 $X$};
    \draw (0,-2) circle [x radius=1, y radius=0.5] node {空間 $Y$};
    \node at (3.5,0) {群$\pi_{1}(X)$};
    \node at (3.5,-2) {群$\pi_{1}(Y)$};
    \node[left] at (0,-1) {ホモトピー同値\hspace{1zw}\rotatebox[origin=c]{-90}{$\simeq$}};
    \node[right] at (3.5,-1) {\rotatebox[origin=c]{-90}{$\simeq$}\hspace{1zw}群同型};
    \draw[->] (1.5,0) -- (2.5,0);
    \draw[->] (1.5,-2) -- (2.5,-2);
  \end{tikzpicture}
\end{center}
となるような空間と群の対応を構成したい。

\sukima \midashi{基本群}

$X$:位相空間
\begin{teigi}[パス]
  \nw{パス(path)} $\defines$ $f \colon [0,1] \longrightarrow X$:連続
\end{teigi}

\begin{teigi}[ホモトピー]
  $\sets{f_{t}}$:パスの\nw{ホモトピー} $\defines$ 次の条件を満たす$\sets{f_{t} \colon [0,1] \longrightarrow X}_{t \in [0,1]}$のこと:
  \begin{itemize}
    \vspace{-0.5\baselineskip}
    \item $\forall t \in [0,1],\quad f_{t}(0) = x_{0},\quad f_{t}(1) = x_{1}$
    \item $F \colon \mapdef{[0,1] \times [0,1]}{X}{(s,t)}{f_{t}(s)}$が連続
  \end{itemize}
\end{teigi}

\begin{teigi}[ホモトープ]
  2つのパス$f', f''$が\nw{ホモトープ} $\defines$ $f', f''$をつなぐホモトピー$\sets{f_{t}}$であって,
  $f_{0} = f'$,$f_{1} = f''$であるものが存在する

  このことを$f' \simeq f''$と書く。
\end{teigi}

\begin{prop}
  ホモトープの$\simeq$は同値関係である。
\end{prop}

\begin{proof}
  パスのホモトープは普通のホモトープの特別な場合であるから,そのときの同値性から従う。
\end{proof}

\begin{teigi}[パスの積(合成)]
  $f,g \colon [0,1] \longrightarrow X$:パス,$f(1) = g(0)$を満たすとする。
  パスの\nw{合成} $f \cdot g \colon [0,1] \longrightarrow X$を次で定義する。
  \begin{equation}
    f \cdot g (s) := \begin{cases}
      f(2s) & \left( 0 \le s \le \dfrac{1}{2} \right) \\
      g(2s-1) & \left( \dfrac{1}{2} \le s \le 1 \right)
    \end{cases}
  \end{equation}
\end{teigi}

\sukima \midashi{注意.} $f' \simeq f$,$g' \simeq g$ならば$f' \cdot g' \simeq f \cdot g$である。

なぜならば,$f$と$f'$を結ぶホモトピー$\sets{f_{t}}$,$g$と$g'$を結ぶホモトピー$\sets{g_{t}}$に対して
$\left( f \cdot g \right)_{t} := f_{t} \cdot g_{t}$とすれば,これが$f' \cdot g'$と$f \cdot g$を結ぶホモトピーになるから。

\begin{teigi}[ループ]
  $x_{0}$を\nw{基点(basepoint)とするループ(loop)} $\defines$ パス$f \colon [0,1] \longrightarrow X$であって,$f(0) = f(1) = x_{0}$
\end{teigi}

\expandafter\ifx\csname readornot\endcsname\relax
  \end{document}
\fi % 基本群
\expandafter\ifx\csname readornot\endcsname\relax
  \documentclass[uplatex]{jsarticle}
  \usepackage{octopus}
  \usepackage{url}

  \renewcommand{\proofname}{\textsf{証明}}
  \renewcommand{\postpartname}{章}
  \renewcommand{\thesection}{\thepart.\arabic{section}}
  \renewcommand{\thepart}{\arabic{part}}
  \makeatletter\renewcommand{\theequation}{\thesection.\arabic{equation}}\@addtoreset{equation}{section}\makeatother

  \newcommand{\octopuspart}[1]{\newpage\part{#1}\setcounter{section}{0}\vspace{3\baselineskip}}

  \DeclareMathOperator{\dcup}{\dot{\cup}}
  \begin{document}
\fi

\section{被覆空間}

\expandafter\ifx\csname readornot\endcsname\relax
  \end{document}
\fi % 被覆空間
\expandafter\ifx\csname readornot\endcsname\relax
  \documentclass[uplatex]{jsarticle}
  \usepackage{octopus}
  \usepackage{url}

  \renewcommand{\proofname}{\textsf{証明}}
  \renewcommand{\postpartname}{章}
  \renewcommand{\thesection}{\thepart.\arabic{section}}
  \renewcommand{\thepart}{\arabic{part}}
  \makeatletter\renewcommand{\theequation}{\thesection.\arabic{equation}}\@addtoreset{equation}{section}\makeatother

  \newcommand{\octopuspart}[1]{\newpage\part{#1}\setcounter{section}{0}\vspace{3\baselineskip}}

  \DeclareMathOperator{\dcup}{\dot{\cup}}
  \begin{document}
\fi

\section{ホモロジー}

\expandafter\ifx\csname readornot\endcsname\relax
  \end{document}
\fi % ホモロジー
\expandafter\ifx\csname readornot\endcsname\relax
 \documentclass{jsarticle}
 \begin{document}
\fi

\section{ホモロジーの計算}

\expandafter\ifx\csname readornot\endcsname\relax
  \end{document}
\fi % ホモロジーの計算
\expandafter\ifx\csname readornot\endcsname\relax
 \documentclass{jsarticle}
 \begin{document}
\fi

\section{特異ホモロジー}

\expandafter\ifx\csname readornot\endcsname\relax
  \end{document}
\fi % 特異ホモロジー

\octopuspart{テンソル}
座標変換に対する不変性

\expandafter\ifx\csname readornot\endcsname\relax
  \documentclass[uplatex]{jsarticle}
  \usepackage{octopus}
  \usepackage{url}

  \renewcommand{\proofname}{\textsf{証明}}
  \renewcommand{\postpartname}{章}
  \renewcommand{\thesection}{\thepart.\arabic{section}}
  \renewcommand{\thepart}{\arabic{part}}
  \makeatletter\renewcommand{\theequation}{\thesection.\arabic{equation}}\@addtoreset{equation}{section}\makeatother

  \newcommand{\octopuspart}[1]{\newpage\part{#1}\setcounter{section}{0}\vspace{3\baselineskip}}

  \DeclareMathOperator{\dcup}{\dot{\cup}}
  \begin{document}
\fi

\section{双対空間}

\expandafter\ifx\csname readornot\endcsname\relax
  \end{document}
\fi % 双対空間
\expandafter\ifx\csname readornot\endcsname\relax
 \documentclass{jsarticle}
 \begin{document}
\fi

\section{テンソルの定義}

\expandafter\ifx\csname readornot\endcsname\relax
  \end{document}
\fi % テンソルの定義
\expandafter\ifx\csname readornot\endcsname\relax
 \documentclass{jsarticle}
 \begin{document}
\fi

\section{反変性・共変性・スカラー・混合テンソル}

\expandafter\ifx\csname readornot\endcsname\relax
  \end{document}
\fi % 反変性・共変性・スカラー・混合テンソル
\expandafter\ifx\csname readornot\endcsname\relax
  \documentclass[uplatex]{jsarticle}
  \usepackage{octopus}
  \usepackage{url}

  \renewcommand{\proofname}{\textsf{証明}}
  \renewcommand{\postpartname}{章}
  \renewcommand{\thesection}{\thepart.\arabic{section}}
  \renewcommand{\thepart}{\arabic{part}}
  \makeatletter\renewcommand{\theequation}{\thesection.\arabic{equation}}\@addtoreset{equation}{section}\makeatother

  \newcommand{\octopuspart}[1]{\newpage\part{#1}\setcounter{section}{0}\vspace{3\baselineskip}}

  \DeclareMathOperator{\dcup}{\dot{\cup}}
  \begin{document}
\fi

\section{交代テンソル・擬テンソル・テンソル密度・Eddintonのε}

\expandafter\ifx\csname readornot\endcsname\relax
  \end{document}
\fi % 交代テンソル・擬テンソル・テンソル密度・Eddintonのε

\end{document}
