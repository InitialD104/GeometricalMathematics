\expandafter\ifx\csname readornot\endcsname\relax
  \documentclass[uplatex]{jsarticle}
  \usepackage{octopus}
  \usepackage{url}

  \renewcommand{\proofname}{\textsf{証明}}
  \renewcommand{\postpartname}{章}
  \renewcommand{\thesection}{\thepart.\arabic{section}}
  \renewcommand{\thepart}{\arabic{part}}
  \renewcommand{\restriction}[2]{\left. #1 \right|_{#2}}
  \makeatletter\renewcommand{\theequation}{\thesection.\arabic{equation}}\@addtoreset{equation}{section}\makeatother

  \newcommand{\octopuspart}[1]{\newpage\part{#1}\setcounter{section}{0}\vspace{3\baselineskip}}

  \DeclareMathOperator{\dcup}{\dot{\cup}}
  \begin{document}
\fi

\section{ホモトピー}

\midashi{2つの空間が「同じ形」をしているとはどういうことか?}

$\xrightarrow{\text{ans}}$位相空間$X,Y$が位相同型 $\defines$ $\exists f \colon X \longrightarrow Y$:連続全単射,$f^{-1}$:連続

例:ドーナツとコーヒーカップ

$\xrightarrow{\text{ans}}$連続変形で移り合う

例:太字のAと細字のAと丸 %図

例:ディスクと一点(明らかにこの2つは位相同型ではない)

このような「空間の連続変形」を定式化したい

\sukima \midashi{\large 変形レトラクション}

$X$:位相空間,$A \subseteq X$とする。

\begin{teigi}[変形レトラクション]
  $X$から$A$への\nw{変形レトラクション(deformation retraction)}とは,
  次の条件を満たす$\sets{f_{t} \colon X \longrightarrow X}_{t \in [0,1]}$のこととする。:
  \begin{itemize}
    \vspace{-0.5\baselineskip}
    \item $f_{0} = \mathrm{id}_{X}$
    \item $f_{1}(X) = A$
    \item $\restriction{f_{t}}{A} = \mathrm{id}_{A}$($\forall t \in [0,1]$)
    \item $\renewcommand{\arraystretch}{0.8} \begin{array}{c@{\:}c@{\:}c@{\:}c}
               & X \times [0,1] & \longrightarrow & X \\
      F \colon & \vin           &                 & \vin \\
               & (x,t)          & \longmapsto     & f_{t}(x)
    \end{array} \renewcommand{\arraystretch}{1.3}$は連続写像
  \end{itemize}
\end{teigi}

\expandafter\ifx\csname readornot\endcsname\relax
  \end{document}
\fi