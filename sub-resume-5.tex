\expandafter\ifx\csname readornot\endcsname\relax
  \documentclass[uplatex]{jsarticle}
  \usepackage{octopus}
  \usepackage{url}

  \renewcommand{\proofname}{\textsf{証明}}
  \renewcommand{\postpartname}{章}
  \renewcommand{\thesection}{\thepart.\arabic{section}}
  \renewcommand{\thepart}{\arabic{part}}
  \renewcommand{\restriction}[2]{\left. #1 \right|_{#2}}
  \makeatletter\renewcommand{\theequation}{\thesection.\arabic{equation}}\@addtoreset{equation}{section}\makeatother

  \newcommand{\octopuspart}[1]{\newpage\part{#1}\setcounter{section}{0}\vspace{3\baselineskip}}

  \DeclareMathOperator{\dcup}{\dot{\cup}}
  \begin{document}
\fi

\section{ホモトピー}

\midashi{2つの空間が「同じ形」をしているとはどういうことか?}

\noindent $\xrightarrow{\text{ans}}$位相空間$X,Y$が位相同型 $\defines$ $\exists f \colon X \longrightarrow Y$:連続全単射,$f^{-1}$:連続

例:ドーナツとコーヒーカップ

\noindent $\xrightarrow{\text{ans}}$連続変形で移り合う

例:太字のAと細字のAと円 %図

例:ディスクと一点(明らかにこの2つは位相同型ではない)

このような「空間の連続変形」を定式化したい

\sukima \midashi{\large 変形レトラクション}

$X$:位相空間,$A \subseteq X$とする。

\begin{teigi}[変形レトラクション]
  $X$から$A$への\nw{変形レトラクション(deformation retraction)}とは,
  次の条件を満たす$\sets{f_{t} \colon X \longrightarrow X}_{t \in [0,1]}$のこととする。:
  \begin{itemize}
    \vspace{-0.5\baselineskip}
    \item $f_{0} = \mathrm{id}_{X}$
    \item $f_{1}(X) = A$
    \item $\restriction{f_{t}}{A} = \mathrm{id}_{A}$($\forall t \in [0,1]$)
    \item $F \colon \mapdef{X \times [0,1]}{X}{(x,t)}{f_{t}(x)}$は連続写像
  \end{itemize}
\end{teigi}

\begin{rei}
  $X \subseteq \mathbb{R}^{n}$:凸集合 i.e. $\forall x,y \in X, \quad \forall \lambda \in [0,1], \quad (1-\lambda) x + \lambda y \in X$とする。
  $z \in X$として
  \begin{equation}
    f_{t} \colon \mapdef{X}{X}{x}{(1-t) x + t z} ,\qquad(t \in [0,1])
  \end{equation}
  は,$X$から$\sets{z}$への変形レトラクションである。(凸性のおかげで写像の行き先が常に$X$に入る)
\end{rei}

\begin{rei}
  $X = \mathbb{R}^{n} \setminus \sets{0}$とする。
  \begin{equation}
    f_{t} \colon \mapdef{X}{X}{x}{(1-t)x+t\dfrac{x}{\left\| x \right\|}}
  \end{equation}
  は$\mathbb{R}^{n} \setminus \sets{0}$から$S^{n-1}$への変形レトラクションである。
\end{rei}

\midashi{問題.} $A \subseteq \mathbb{R}^{n}$:閉凸集合として,$\mathbb{R}^{n}$から$A$への変形レトラクションを構成せよ。
({\bf ヒント}:近接点写像)

\sukima \midashi{\large ホモトピー}

$X,Y$:位相空間

\begin{teigi}[ホモトピー]
  $\sets{f_{t}}$:\nw{ホモトピー(homotopy)} $\defines$ 次の条件を満たす$\sets{f_{t} \colon X \longrightarrow Y}_{t \in [0,1]}$のこと:
  \begin{itemize}
    \vspace{-0.5\baselineskip}
    \item $F \colon \mapdef{X \times [0,1]}{Y}{(x,t)}{f_{t}(x)}$が連続
  \end{itemize}

  また,$g \colon X \longrightarrow Y$と$h \colon X \longrightarrow Y$をつなぐホモトピー$\sets{f_{t}}$とは
  上の$\sets{f_{t}}$に関する条件に加えて,$f_{0} = g$,$f_{1} = h$を満たすもののことを指す。
  このとき,$g$と$h$は\nw{ホモトープ(homotope)}または\nw{ホモトピック(homotopic)}といい,$g \simeq h$と書く。
\end{teigi}

\begin{hodai}
  ホモトープの関係$\simeq$は同値関係
\end{hodai}

\begin{proof}
  \begin{itemize}
    \vspace{-0.5\baselineskip}
    \item $g \simeq g$は$f_{t} \equiv g$とすればよい。
    \item $g \simeq h$とすると,このときのホモトピー$f_{t}$に対して$f_{1-t}$を考えれば$h \simeq g$である。
    \item $f \simeq g$かつ$g \simeq h$とする。$f$と$g$を結ぶホモトピーを$\sets{f_{t}}$,$g$と$h$を結ぶホモトピーを$\sets{g_{t}}$とする。
    このとき,$f$と$h$の間に写像の族$\sets{\overline{f}_{t}}$を
    \begin{equation}
      \overline{f}_{t} = \begin{cases}
        f_{2t} & (0 \le t \le 1/2) \\
        g_{2t-1} & (1/2 \le t \le 1)
      \end{cases}
    \end{equation}
    で定める。すると,$\overline{f}_{0} = f$であり$\overline{f}_{1} = h$である。
    このとき,
    \begin{equation}
      \overline{F} \colon \mapdef{X \times [0,1]}{Y}{(x,t)}{\overline{f}_{t}(x)}
    \end{equation}
    は$X \times [0,1/2]$上と$X \times [1/2,1]$上のそれぞれで連続であり,この2つの集合はともに閉集合である。
    \rref{補題}{close.continuous}から, $f$は$X \times [0,1]$上でも連続であり,$f$と$h$を結ぶホモトピーになっている。
  \end{itemize}
\end{proof}

上の補題の証明に用いた事実を明記して証明する。:

\begin{hodai}
  \label{close.continuous}
  $X = A \cup B$,$A,B$:閉とする。$f \colon X \longrightarrow Y$に対して
  $\restriction{f}{A},\restriction{f}{B}$がそれぞれ連続であるならば$f$は$X$上で連続である。
\end{hodai}

\begin{proof}
  $F \subseteq Y$:閉とすると,$f^{-1} (F) = \restriction{f^{-1}}{A}(F) \cup \restriction{f^{-1}}{B}(F)$は有限個の閉集合のunionなので閉集合である。
\end{proof}

\begin{teigi}[ホモトピー類]
  ホモトピーの同値関係による同値類を\nw{ホモトピー類}という。
\end{teigi}

\sukima \midashi{注意.} 変形レトラクションは,恒等写像$\mathrm{id}_{X}$とレトラクション$r$を結ぶホモトピーである。
ここで,\nw{レトラクション}とは次を満たす写像$r \colon X \longrightarrow X$のことである。
\begin{itemize}
  \vspace{-0.5\baselineskip}
  \item $r(X) = A$
  \item $\restriction{r}{A} = \mathrm{id}_{A}$
\end{itemize}

\sukima \midashi{注意.} $Y \subseteq \mathbb{R}^{n}$のとき,$g \colon X \longrightarrow Y$,$h \colon X \longrightarrow Y$を結ぶホモトピーとして,
$(1-t)g+th$(これを\nw{線形ホモトピー}という)を作りたくなるが,
例えば凸性が保証されていないときには$(1-t)g(x) + t h(x) \notin Y$となる可能性があるので気をつける必要がある。

\sukima \midashi{注意.} もし$Y$が弧状連結であれば,$g(x)$と$h(x)$を結ぶパスに沿っていつもホモトピーを作ることができるのだろうか。
例えば$X = Y = S^{1}$として,$\mathrm{id}_{S^{1}}$と$S^{1}$上の一点へのレトラクションを結ぶホモトピーは{\bf できなさそう}である。

\sukima \midashi{\large ホモトピー同値}

\begin{teigi}[ホモトピー同値]
  位相空間$X,Y$が\nw{ホモトピー同値} $\defines$ $\exists f \colon X \longrightarrow Y$:連続,$\exists g \colon Y \longrightarrow X$:連続,
  $f \circ g \simeq \mathrm{id}_{Y}$,$g \circ f \simeq \mathrm{id}_{X}$

  また,このとき$X$と$Y$は\nw{同じホモトピー型をもつ}ともいう。ホモトピー同値であることを$X \simeq Y$と書く。
\end{teigi}

\midashi{注意.} 位相同型ならばホモトピー同値である。

\begin{hodai}
  ホモトピー同値の関係$\simeq$は同値関係
\end{hodai}

\begin{proof}
  ここでは推移律だけを示す。$X \simeq Y$かつ$Y \simeq Z$とする。
  \begin{align}
    &\exists f \colon X \longrightarrow Y, \quad \exists g \colon Y \longrightarrow X, \quad f \circ g \simeq \mathrm{id}_{Y}, \quad g \circ f \simeq \mathrm{id}_{X} \\
    &\exists f' \colon Y \longrightarrow Z, \quad \exists g' \colon Z \longrightarrow Y, \quad f' \circ g' \simeq \mathrm{id}_{Z}, \quad g' \circ f' \simeq \mathrm{id}_{Y}
  \end{align}
  であるから,$\overline{f} = f' \circ f$,$\overline{g} = g \circ g'$と定めれば,これは連続写像であり,
  $\overline{f} \circ \overline{g} = f' \circ f \circ g \circ g' \simeq f' \circ g' \simeq \mathrm{id}_{Z}$
  である。同様に,$\overline{g} \circ \overline{f} \simeq \mathrm{id}_{X}$である。したがって,$X \simeq Z$である。
\end{proof}

\expandafter\ifx\csname readornot\endcsname\relax
  \end{document}
\fi