\expandafter\ifx\csname readornot\endcsname\relax
  \documentclass[uplatex]{jsarticle}
  \usepackage{octopus}
  \usepackage{url}

  \renewcommand{\proofname}{\textsf{証明}}
  \renewcommand{\postpartname}{章}
  \renewcommand{\thesection}{\thepart.\arabic{section}}
  \renewcommand{\thepart}{\arabic{part}}
  \makeatletter\renewcommand{\theequation}{\thesection.\arabic{equation}}\@addtoreset{equation}{section}\makeatother

  \newcommand{\octopuspart}[1]{\newpage\part{#1}\setcounter{section}{0}\vspace{3\baselineskip}}

  \DeclareMathOperator{\dcup}{\dot{\cup}}
  \begin{document}
\fi

\section{連結性}
$(X, \mathcal{O})$を位相空間とする。

\sukima \midashi{復習}(省略)

\begin{teigi}[連結]
  \begin{flushleft}
    \begin{tabular}{llll}
      $X$:\nw{連結} & $\defines$ & $\mathcal{O} \cap \mathfrak{A} = \sets{\emptyset, X}$ & ($X$上開かつ閉なる集合は$\emptyset$,$X$のみ)\\
      $A \subseteq X$:連結 & $\defines$ & $A$が$X$の部分位相空間として連結である
    \end{tabular}
  \end{flushleft}
\end{teigi}

すなわち,$X$が連結{\bf でない}ことが次の二つの同値な条件で特徴づけられる。:
\begin{center}
  $\exists O \in \mathcal{O} \colon \quad O \neq \emptyset, \quad X \setminus O \in \mathcal{O}$ \\
  $X$は非交で非空な開集合の和集合に分解される
\end{center}

\begin{prop}
  $f \colon X \longrightarrow Y$:連続,$A \subseteq X$:連結のとき,$f(A)$:連結
\end{prop}

「連結性」は連続写像によって「不変」な性質であり,
位相的な性質であるといえる。(同相写像によって不変)

\begin{proof}
  $B \subseteq f(A)$を開かつ閉な集合とする。
  \begin{align}
    & \exists G: Y\text{上開}, \quad \exists F: Y\text{上閉}, \quad B = G \cap f(A) = F \cap f(A) \\
    \to & f^{-1}(B) = f^{-1} (G) \cap f^{-1}(f(A)) = f^{-1}(F) \cap f^{-1}(f(A)) \\
    \to & f^{-1}(B) \cap A = f^{-1}(G) \cap A = f^{-1}(F) \cap A
  \end{align}
  $f$の連続性より$f^{-1}(G)$:開,$f^{-1}(F)$:閉。
  よって,相対位相の定義より,$f^{-1}(G) \cap A$:$A$上開,$f^{-1}(F) \cap A$:$A$上閉。
  $f^{-1}(B) \cap A$は$A$の相対位相で開かつ閉。
  $A$の連結性より$f^{-1}(B) \cap A = \emptyset \: \text{or} \: A$である。

  $f^{-1} (B) \cap A = \emptyset$なら$B = \emptyset$,
  $f^{-1} (B) \cap A = A$なら$f(f^{-1}(B)) \cap f(A) = f(A)$であるから$B = f(A)$である。
  つまり,$f(A)$上で開かつ閉な集合は$\emptyset$か$f(A)$であるので,$f(A)$は連結である。
\end{proof}

\begin{prop}
  $A, B \subseteq X$に対し,$A \subseteq B \subseteq \overline{A}$であるとする。
  このとき,$A$:連結$\Longrightarrow B$:連結である。
  特に,連結な集合の閉包は連結である。
\end{prop}

\begin{proof}
  $B' \subseteq B$:$B$上開かつ閉とする。
  \begin{equation}
    \exists G: X\text{上開},\quad \exists F: X\text{上閉}, \quad B' = G \cap B = F \cap B
  \end{equation}
  であり,$A \subseteq B$より
  $A \cap B' = G \cap A = F \cap A$である。よって,$A \cap B'$は$A$上開かつ閉である。
  よって,$A$の連結性から$A \cap B' = \emptyset \: \text{or} \: A$である。

  $A \cap B' = \emptyset$のとき,$G \cap A = \emptyset$であり,$G \cap \overline{A} = \emptyset$,よって,$B' = \emptyset$である。

  $A \cap B' = A$のとき,$F \supseteq A$であるから,$F \supseteq \overline{A} \supseteq B$であり,$B' = F \cap B = B$である。

  これより$B$の連結性が従う。
\end{proof}

\begin{prop}
  \label{prop:renketu.union}
  $A,B$:連結,$A \cap B \neq \emptyset$とする。このとき$A \cup B$:連結である。
\end{prop}

\begin{proof}
  $N \subseteq A \cup B$:$A \cup B$上開かつ閉とする。
  \begin{equation}
    \exists G: X\text{上開},\quad \exists F: X\text{上閉}, \quad N = G \cap (A \cup B) = F \cap (A \cup B)
  \end{equation}
  特に$N \cap A = G \cap A = F \cap A$であり,$N \cap A$は$A$上開かつ閉。
  $N \cap B = G \cap B = F \cap B$であり,$N \cap B$は$B$上開かつ閉。
  よって,$N \cap A = \emptyset \: \text{or} \: A$,$N \cap B = \emptyset \: \text{or} \: B$である。

  もし$N \cap A = \emptyset$ならば$A \cap B \neq \emptyset$より$N \cap B = B$となることはない。よって,$N = \emptyset$である。

  もし$N \cap A = A$ならば$N \supseteq A \supseteq A \cap B \neq \emptyset$より$N \cap B = B$である。よって,$N = A \cup B$である。
\end{proof}

同様に次も成立する。

\begin{prop}
  \label{prop:renketu.union.gen}
  $\forall \lambda \in \Lambda, A_{\lambda}$:連結,
  ${\displaystyle \bigcap_{\lambda \in \Lambda} A_{\lambda} \neq \emptyset}$とする。
  このとき,${\displaystyle \bigcup_{\lambda \in \Lambda} A_{\lambda}}$:連結である。
\end{prop}

\begin{teigi}[連結成分]
  関係$\sim$を
  \begin{equation}
    x \sim y \quad \defines \quad \exists A \subseteq X \text{:連結}, \quad x \in A, \quad y \in A
  \end{equation}
  で定めるとこれは同値関係である。この同値関係による同値類を\nw{連結成分}という。
  $C_{x}$で$x$を含む連結成分を表すとする。
\end{teigi}

\begin{proof}
  \midashi{$\sim$が同値関係をなすこと}

  $\sets{x}$は連結であるから,$x \sim x$である。
  また,明らかに$x \sim y$ならば$y \sim x$である。
  $x \sim y$かつ$y \sim z$とする。このとき,ある連結な集合$A,B$が存在して,
  $x, y \in A$かつ$y, z \in B$である。いま,$y \in A \cap B \neq \emptyset$であるから,
  \rref{命題}{prop:renketu.union}より$A \cup B$は連結である。
  $x, z \in A \cup B$であるから,$x \sim z$である。
\end{proof}

\begin{prop}
  $C_{x}$は$x$を含む最大の連結部分集合である。
  また,$C_{x}$は閉集合である。
\end{prop}

\begin{proof}
  $A$を$x$を含む連結な集合とする。このとき,すべての$y \in A$に対して$y \sim x$であるから,$A \subseteq C_{x}$である。
  よって,${\displaystyle \bigcup \sets{A | x \in A \text{かつ} Aは連結}} \subseteq C_{x}$であり,結局,
  ${\displaystyle \bigcup \sets{A | x \in A \text{かつ} Aは連結}} = C_{x}$である。
  これより\rref{命題}{prop:renketu.union.gen}から$C_{x}$は連結である。
  よって,$C_{x}$の最大性が従う。また,最大性から
  $\overline{C_{x}} = C_{x}$となり,$C_{x}$は閉集合である。
\end{proof}

\begin{prop}
  $X,Y$:連結とする。このとき直積位相のもとで$X \times Y$:連結である。
\end{prop}

\begin{proof}
  $(x,y), (x', y') \in X \times Y$をとる。$C_{(x,y)} = C_{(x',y')}$を示せばよい。
  いま,$Y$:連結ならば$\sets{x} \times Y$:連結である。
  \begin{quote}
    背理法による。すなわち,$\sets{x} \times Y$が連結でないとする。
    このとき,ある非空非交な$\sets{x} \times Y$上の開集合$A,B$が存在して,$\sets{x} \times Y = A \cup B$である。
    このときさらにある非空な$Y$上の開集合$A',B'$が存在して
    $A = \sets{(x,y) | y \in A'} = \sets{x} \times A'$,
    $B = \sets{(x,y) | y \in B'} = \sets{x} \times B'$である。
    よって,$A' \cup B' = Y$となり$Y$は連結でない。これは矛盾。
  \end{quote}
  同様に$X$:連結ならば$X \times \sets{y'}$:連結である。
  したがって,連結成分を定義したときの同値関係を$\sim$として
  $(x,y) \sim (x, y') \sim (x',y')$となる。よって,$C_{(x,y)} = C_{(x',y')}$である。
\end{proof}

\begin{prop}[中間値の定理]
  $X$:連結,$f \colon X \longrightarrow \mathbb{R}$:連続,
  $x,y \in X$ならば$f(x) < f(y)$とする。
  このとき,
  \begin{equation}
    \forall \alpha \in \mathbb{R}, \quad f(x) < \alpha < f(y) \Longrightarrow \exists z \in X, \quad f(z) = \alpha
  \end{equation}
\end{prop}

\begin{proof}
  背理法による。$f^{-1} (\alpha) = \emptyset$とする。
  $M := f^{-1}((- \infty, \alpha))$は$f$の連続性から開集合である。
  また,$f^{-1}((- \infty, \alpha]) = f^{-1}((- \infty, \alpha)) \cup f^{-1}(\alpha) = f^{-1}((- \infty, \alpha))$は$f$の連続性から閉集合である。
  ここで,$x \in M$であるが$y \notin M$であるので,これは$X$の連結性に矛盾する。
\end{proof}

\begin{hodai}
  $a<b$として,$\left[ a,b \right]$は連結。
\end{hodai}

\begin{proof}
  $O, O'$:開,$O \cap O' \neq \emptyset$,$\left[ a,b \right] = O \cup O'$とする。
  $a \in O$とする。$c = \sup \sets{x| \left[ a, x \right] \subseteq O}$とする。
  このとき,$a < c < b$であり,任意の$\varepsilon > 0$に対して$c - \varepsilon \notin O'$,$c + \varepsilon \notin O$
\end{proof}

\expandafter\ifx\csname readornot\endcsname\relax
  \end{document}
\fi