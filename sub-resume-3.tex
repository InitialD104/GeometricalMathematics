\expandafter\ifx\csname readornot\endcsname\relax
  \documentclass[uplatex]{jsarticle}
  \usepackage{octopus}
  \usepackage{url}

  \renewcommand{\proofname}{\textsf{証明}}
  \renewcommand{\postpartname}{章}
  \renewcommand{\thesection}{\thepart.\arabic{section}}
  \renewcommand{\thepart}{\arabic{part}}
  \makeatletter\renewcommand{\theequation}{\thesection.\arabic{equation}}\@addtoreset{equation}{section}\makeatother

  \newcommand{\octopuspart}[1]{\newpage\part{#1}\setcounter{section}{0}\vspace{3\baselineskip}}

  \DeclareMathOperator{\dcup}{\dot{\cup}}
  \begin{document}
\fi

\section{連結性}
$(X, \mathcal{O})$を位相空間とする。

\sukima \midashi{復習}(省略)

\begin{teigi}[連結]
  \begin{flushleft}
    \begin{tabular}{llll}
      $X$:\nw{連結} & $\defines$ & $\mathcal{O} \cap \mathfrak{A} = \sets{\emptyset, X}$ & ($X$上開かつ閉なる集合は$\emptyset$,$X$のみ)\\
      $A \subseteq X$:連結 & $\defines$ & $A$が$X$の部分位相空間として連結である
    \end{tabular}
  \end{flushleft}
\end{teigi}

すなわち,$X$が連結{\bf でない}ことが次の二つの同値な条件で特徴づけられる。:
\begin{center}
  $\exists O \in \mathcal{O} \colon \quad O \neq \emptyset, \quad X \setminus O \in \mathcal{O}$ \\
  $X$は非交で非空な開集合の和集合に分解される
\end{center}

\begin{prop}
  $f \colon X \longrightarrow Y$:連続,$A \subseteq X$:連結のとき,$f(A)$:連結
\end{prop}

「連結性」は連続写像によって「不変」な性質であり,
位相的な性質であるといえる。(同相写像によって不変)

\begin{proof}
  $B \subseteq f(A)$を開かつ閉な集合とする。
  \begin{align}
    & \exists G: Y\text{上開}, \quad \exists F: Y\text{上閉}, \quad B = G \cap f(A) = F \cap f(A) \\
    \to & f^{-1}(B) = f^{-1} (G) \cap f^{-1}(f(A)) = f^{-1}(F) \cap f^{-1}(f(A)) \\
    \to & f^{-1}(B) \cap A = f^{-1}(G) \cap A = f^{-1}(F) \cap A
  \end{align}
  $f$の連続性より$f^{-1}(G)$:開,$f^{-1}(F)$:閉。
  よって,相対位相の定義より,$f^{-1}(G) \cap A$:$A$上開,$f^{-1}(F) \cap A$:$A$上閉。
  $f^{-1}(B) \cap A$は$A$の相対位相で開かつ閉。
  $A$の連結性より$f^{-1}(B) \cap A = \emptyset \: \text{or} \: A$である。

  $f^{-1} (B) \cap A = \emptyset$なら$B = \emptyset$,
  $f^{-1} (B) \cap A = A$なら$f(f^{-1}(B)) \cap f(A) = f(A)$であるから$B = f(A)$である。
  つまり,$f(A)$上で開かつ閉な集合は$\emptyset$か$f(A)$であるので,$f(A)$は連結である。
\end{proof}

\begin{prop}
  $A, B \subseteq X$に対し,$A \subseteq B \subseteq \overline{A}$であるとする。
  このとき,$A$:連結$\Longrightarrow B$:連結である。
  特に,連結な集合の閉包は連結である。
\end{prop}

\begin{proof}
  $B' \subseteq B$:$B$上開かつ閉とする。
  \begin{equation}
    \exists G: X\text{上開},\quad \exists F: X\text{上閉}, \quad B' = G \cap B = F \cap B
  \end{equation}
  であり,$A \subseteq B$より
  $A \cap B' = G \cap A = F \cap A$である。よって,$A \cap B'$は$A$上開かつ閉である。
  よって,$A$の連結性から$A \cap B' = \emptyset \: \text{or} \: A$である。

  $A \cap B' = \emptyset$のとき,$G \cap A = \emptyset$であり,$G \cap \overline{A} = \emptyset$,よって,$B' = \emptyset$である。

  $A \cap B' = A$のとき,$F \supseteq A$であるから,$F \supseteq \overline{A} \supseteq B$であり,$B' = F \cap B = B$である。

  これより$B$の連結性が従う。
\end{proof}

\begin{prop}
  $A,B$:連結,$A \cap B \neq \emptyset$とする。このとき$A \cup B$:連結である。
\end{prop}

\begin{proof}
  $N \subseteq A \cap B$:$A \cup B$上開かつ閉とする。
  \begin{equation}
    \exists G: X\text{上開},\quad \exists F: X\text{上閉}, \quad N = G \cap (A \cup B) = F \cap (A \cup B)
  \end{equation}
  特に$N \cap A = G \cap A = F \cap A$であり,$N \cap A$は$A$上開かつ閉。
  $N \cap B = G \cap B = F \cap B$であり,$N \cap B$は$B$上開かつ閉。
  よって,$N \cap A = \emptyset \: \text{or} \: A$,$N \cap B = \emptyset \: \text{or} \: B$である。

  もし$N \cap A = \emptyset$ならば$A \cap B \neq \emptyset$より$N \cap B = B$となることはない。よって,$N = \emptyset$である。

  もし$N \cap A = A$ならば$N \supseteq A \supseteq A \cap B \neq \emptyset$より$N \cap B = B$である。よって,$N = A \cup B$である。
\end{proof}

\expandafter\ifx\csname readornot\endcsname\relax
  \end{document}
\fi